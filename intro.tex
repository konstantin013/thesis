\section{Введение}

Суть КАСКО, как и других страхований не жизни, заключается в том, что страховая компания берёт на себя риски клиента взамен за стоимость полиса. Здесь важно наличие случайности риска, причём, как в его размере, так и в факте его наступления. Таким образом, клиент приобретает некоторую определённость сохранности свого имущества. 

В добровольных видах страхования клиент покупает полис, если предложенная за него цена является уместной. И здесь важно дать как можно более точную оценку рискам для данного клиентом. Если оценка будет заниженной, то страховая компания понесёт убытки. Напротив, если завысить цену, то клиент может предпочесть другую страховую компанию.

При этом, важно провести как можно более большую дифференцицию по клиентам. Пусть у нас есть две группы людей $A, B$ с различными ожидаемыми потерями $E_A, E_B$. Тогда для общей группы их ожидаемые потери будут $E: E_A < E < E_B$. если эти числа будут существенно различаться, то группа A, возможно, предпочтёт выбрать другую страховую компанию. Это означает, что на самом деле ожидаемые потери $E_B$, а мы их оцениваем, как  $E$. Это может привести к неприятным последствиям. Данный пример показывает, что для тарификации необходимо использовать как больше параметров.