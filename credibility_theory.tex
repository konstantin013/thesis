\subsection{Использование теории достоверности в многоуровневых факторах}
	Рассмотрим многоуровневый фактор $F$ с уровнями $\{ 1, 2, ..., J\}$. Будем обозначать наблюдения рассматриваемой случайной величины, как $Y_{jt}$, где $j$ - уровень фактора $F$, $t$ - номер среди таких наблюдений. $\mu$ - среднее по всем наблюдениям:
	
$$
	\mu = \frac{\sum_{jt}w_{jt}Y_{jt}}{\sum_{jt}w_{jt}}
$$

Для каждого уровня $j \in \{1, ..., J\}$ посчитаем для него среднее:

$$
	\overline{Y}_j = \frac{\sum_{t}w_{jt}Y_{jt}}{\sum_{jt}w_{jt}}
$$

Здесь, в отличии от обычного фактора, метод максимального правдоподобия неприменим, т.к. каждый из коэффициентов $e^{\beta_j}$ будет рассчитываться по слишком малому количеству наблюдений.

Для каждого уровня $j$ будем делать разумное предсказание на реальное среднее наблюдаемой случайной величины. Здесь нужно найти некоторый компромисс между $\overline{Y}_j$ и $\mu$. Первый является нестабильным, поскольку посчитан на малых данных. Второй стабильный, но никак не отражает зависимости от уровня $j$.

Будем считать, что каждый уровень подвержен \textit{случайному эффекту} $U_j$. Для мультипликативной модели имеем:
$$
	E[Y_{jt} | U_j] = \mu U_j
$$
,где $E[U_j] = 1$

Для удобства сделаем замену $V_j = \mu U_j$. Тогда имеем:
$$
	E[Y_{jt} | V_j] = V_j
$$
,где $E[V_j] = \mu$

Здесь необходимо сделать предположения
\begin{itemize}
	\item $\forall j$ случайные векторы $(Y_{jt}, V_j)$ независимы
	\item $V_j$ одинаково распределены со средним $E[V_j] = \mu > 0$
	\item $\forall j$ все $Y_{jt}$ при условии $V_j$ независимы со средним $E[Y_{jt} | V_j] = V_j$
\end{itemize}
