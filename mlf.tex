

\subsection{Определение MLF}

Для начала, стоит определиться, что какие переменные мы рассматриваем как multi-level. 
Мы будем называть фактор multi-level, если он имеет относительно большое количество уровней, многие из которых не имеют много данных, при этом, уровни фактора не имеют никакого естественного порядка. 
приведём несколько примеров:



\begin{itemize}
	\item Населённый пункт. Является одним из самых важных факторов. Через него можуг внести эффект такие показатели, например, как качество дорог, по которым ездит страхователь, качество жизни горожан. Такие показатели имеют сильное влияние учет сильно повышает качество модели. Но здесь мы сталкиваемся с проблемой. В состав Российской Федерации входит более тысячи городов. При этом лишь по нескольким из них имеется достаточная статистика. Например, такие риски, как вероятность угона или тотальной гибели автомобиля сильно зависят от фактора населённый пункт. Но поскольку эти частоты имеют порядок 0.001, то для их учёта необходима статистика хотя бы несколько тысячь полисо-лет. Понятно, что далеко не каждый населённый пункт имеет такую статистическую базу. В этом примере показана основная проблема данной работы. С одной стороны, мало какие населённые пункты пункты имеют достаточно данных для учёта их как фактора. Но с другой стороны, если мы видим, что какой-то город с небольшой статистикой имеет очень большое влияние, скажем в нём частота ДТП Виновник намного больше чем в остальных, то не учитывать это влияние совсем было бы неправильно. 
	Можно было бы объеденить все небольшие города в один из одного региона или даже федерального округа для избежания небольших объёмов данных, но это приводит к другим проблемам: населённый пункт с большим средним доходом было бы неправильно объединять с неблагополучным соседним. Было бы разумнее объединить его  с похожим населённым пунктом из другого региона. Поэтому этот фактор остаётся многоуровневым.
	\item Модель автомобиля. Это один из самых важных факторов в тарификации тоже. Здесь возникают проблемы схожие с населённым пунктом. Какие-то модели более привлекательны для угона, какие-то редкие модели дороже ремонтировать при обычных ДТП, или владельцы каких-то марок более склонны к тем или иным ДТП. Очевидно, уровни фактора не образуют никакого естественного порядка. При этом их очень много и лишь несколько самых популярных из них имеют не слишком маленькую долю.

\end{itemize}
	В примере с населённым пунктом была поставлена основная проблема. Есть некоторые уровни многоуровневого фактора, которые неправильно было бы учитывать 