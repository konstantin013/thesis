
\subsection{Введение в ОЛМ}


Назначение обобщенных линейных моделей (ОЛМ) показать зависимость между наблюдаемой величиной, также называемой откликом $Y$ и зависимыми переменными, называемыми факторами $X$. Модель рассматривает наблюдения $Y_i$, как реализации случайной величины $Y$.
%translate_problem covariate - фактор

Как и в многих других методах машинного обучения стоит отдельно рассмотреть категориальные и вещественные факторы. Категориальные признаки принимают конечное число неупорядоченных значений, называемых, уровнями факторами. Примерами могут служить пол человека, марка или модель его автомобиля, город проживания, взят ли автомобиль в кредит и т.д. Вещественные факторы принимают числовые значения, например, мощность, стоимость автомобиля, стаж, возраст водителя, число водителей,прописанных в полисе и т.д.


\subsection{ОЛМ на практике}

Нашей конечной задачей является оценить ожидаемые риски с данного полиса по имеющимся данным. Для использования ОЛМ важно разделить риски следующим образом:

\begin{itemize}
	\item по виду риска. А именно:
		\begin{itemize}
			\item Тотальная гибель автомобиля
			\item Угон
			\item ДТП Виновник
			\item ДТП Потерпевший
			\item Противоправные действия третих лиц
		\end{itemize}
		
		это очень важное действие, поскольку факторы сказываются на этих рисках совершенно по-разному. Например, можно утверждать, что водители с небольшим стажем имеют большие риски стать виновником в ДТП, но это нельзя говорить о, скажем, Противоправных действиях третих лиц или угонах. 
	
	\item разделение на частоту и тяжесть. 
		для каждого полиса отдельно стоит оценивать ожидаемую частоту по данному риску и ожидаемую тяжесть (при условии, что происшествие произошло) по той же простой причине: зависимости частоты и тяжести имеют совершенно разную природу. Самые наглядные примеры это риски Тотальная гибель автомобиля и Угон. Тяжесть у этих рисков не зависят ни от чего, кроме страховой суммы, а вот про частоту этих рисков, разумеется этого сказать нельзя.
\end{itemize}




\subsection{Структура ОЛМ}

Структура ОЛМ может быть описана, как:
\[
	\mu_i = E[Y_i] = g^{-1}(\sum_{j}{X_{ij}} \beta_{j} + \xi_{i})		
\]

\[
	Var[Y_i] = \frac{\phi V(\mu_i)}{\omega_i}
\]

где

\begin{itemize}
	\item $Y$ вектор наблюдений
	\item $g(x)$ функция связи - заданая обратимая функия, которая связывает отклик с линейной комбинацией признаков.
	%translate_problem design matrix - проекционная матрица
	\item $X$ - проекционная матрица, полученная из факторов
	\item $\beta$ - вектор параметров модели, которые будут оцениваться.
	\item $\xi$ - вектор известных смещений.
	\item $\phi$ - параметр для масштабирования функции $V(x)$
	\item $V(x)$ - функция дисперсии
	\item $w$ - вектор весов, которые задают достоверность данных для каждого наблюдения.	
\end{itemize}

Вектор откликов, проекционная матрица и смещения, основываются на данных. Функция связи $g(x)$, функция дисперсии $V(x)$ и параметр $\phi$ задаются исследователем.
