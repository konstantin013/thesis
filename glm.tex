
\subsection{Введение в ОЛМ}


Назначение обобщенных линейных моделей (ОЛМ) показать зависимость между наблюдаемой величиной, также называемой откликом $Y$ и зависимыми переменными, называемыми факторами $X$. Модель рассматривает наблюдения $Y_i$, как реализации случайной величины $Y$.
%translate_problem covariate - фактор

Как и в многих других методах машинного обучения стоит отдельно рассмотреть категориальные и вещественные факторы. Категориальные признаки принимают конечное число неупорядоченных значений, называемых, уровнями факторами. Примерами могут служить пол человека, марка или модель его автомобиля, город проживания, взят ли автомобиль в кредит и т.д. Вещественные факторы принимают числовые значения, например, мощность, стоимость автомобиля, стаж, возраст водителя, число водителей,прописанных в полисе и т.д.

Случайную величину Y будем представлять в виде суммы её математического ожидания $\mu$ и случайной величины $\varepsilon$
\[
	Y = \mu + \varepsilon
\]
В дальнейшем будет показано, что с помощью факторов предсказывается $\varepsilon$, а это предсказание оценивается на основе распределения случайной величины $\varepsilon$.





\subsection{Предположения обобщенной линейной модели}


\begin{itemize}
	\item Элементы отклика $Y$ являются независимыми. При этом они относятся к семейству экспоненциально распределенных 
	\item Влияние объясняющих переменных на отклик является аддитивным, но только после некоторого преобразования,которое выражается \textit{функцией связи} 


\end{itemize}
		




\subsection{ОЛМ на практике}

Нашей конечной задачей является оценить ожидаемые риски с данного полиса по имеющимся данным. Для использования ОЛМ важно разделить риски следующим образом:

\subsubsection{по виду риска}. А именно:
		\begin{itemize}
			\item Тотальная гибель автомобиля
			\item Угон
			\item ДТП Виновник
			\item ДТП Потерпевший
			\item Противоправные действия третих лиц
			\item Остальное
			
		\end{itemize}
		
		это очень важное действие, поскольку многие факторы сказываются на этих рисках совершенно по-разному. Например, можно утверждать, что водители с небольшим стажем имеют намного большие риски стать виновником в ДТП, но это нельзя говорить о, скажем, Противоправных действиях третих лиц или угонах. И наоборот, на такие риски большое флияние оказывает такой фактор, как наличие сигнализации, но на модель частоты виновника ДТП он не влияет.
	
\subsubsection{разделение на частоту и тяжесть.} 
		для каждого полиса отдельно стоит оценивать ожидаемую частоту по данному риску и ожидаемую тяжесть (при условии, что происшествие произошло) по той же простой причине: зависимости частоты и тяжести имеют совершенно разную природу. Самые наглядные примеры это риски Тотальная гибель автомобиля и Угон. Тяжесть у этих рисков не зависят ни от чего, кроме страховой суммы, а вот про частоту этих рисков, разумеется этого сказать нельзя. 
		
		Обозначим $N$ - случайная величина, число убытков для данного полиса, а $Z_1, Z_2, ... , Z_N$ - случайные величины, тяжести этих убытков. Здесь делается разумное предположение, что случайные величины $N, Z_1, Z_2, ... , Z_N$ являются независимыми, а $Z_1, Z_2, ... , Z_N$ одинакого распределёнными.Тогда суммарный ущерб для данного полиса составит 

$$ S = Z_1 + Z_2 + ... + Z_N $$

Тогда можно заметить, что

$$ E[S] = E[Z_1 + Z_2 + ... + Z_N] = E[E[Z_1 + Z_2 + ... + Z_N] | N] $$

$$ E[S] = E[E[Z_1 | N] + ... + E[Z_N | N]] = E[N E[Z]] $$

$$ E[S] = E[N]  E[Z] $$

Также необходимо учесть зависимость $N$ от продолжительности полиса. Здесь стоит сделать ещё одно предположение на $N(t)$ - число убытков ко времени о том, что $N(t)$ - стационарный процесс с незывисимыми приращениями, что является весьма естественным предположением. Отсюда следует, что $E[N]$ пропорционально продолжительности полиса. Из сказанного следует, что мы можем отдельно моделировать частоту и тяжесть убытков, а затем просто перемножить их результаты и продолжительность полиса. 



\subsubsection{проблемы связанные с маленькими группами}

Мы покажем, почему все уровни любого фактора не должны содержать мало данных. Рассмотрим модели частот. Здесь же мы и поясним, что в данном случае означает мало. Предположим, что у нас имееться некоторый фактор $F$, у которого есть уровень $L$, которому удовлетворяет мало данных из статистической базы. А раз этот уровень содержит мало данных, то велика вероятность того, что в данных с уровнем $L$ фактора $F$  нет ни одного происшествия. Это особенно актуально для рисков с маленькими частотами, а именно угоны и Тотали. Для них частоты порядка 0.001. А поскольку частота ДТП будет предсказываться, как

$$ 
exp\{ \sum_{j=1}^p X_{ip} \beta_p   +   \xi_i \}
$$. 


для всех $i$. Тогда для всех наблюдений $i$, которые относятся к фактору $L$ будет следующая оценка

$$ 
0 = exp\{ \sum_{j=1}^p X_{ip} \beta_p   +   \xi_i \} =
$$. 

пусть коэффицент $\beta_j$ соответствует уровню $L$ фактора $F$. Тогда можно сделать следующее наблюдение: 

\textbf{Утверждение}. Для любого вектора $\beta$ при уменьшении компоненты $\beta_j$, функция правдоподобия будет уменьшаться.

Действительно, при уменьшении компоненты $\beta_j$, правдоподобие на наблюдениях $i$ не относящихся к уровню $L$ не измениться т.к. по построению матрицы плана соотвествующий коэффицент $X_{ij}$ , будет 0, а значит выражение 
$$ 
exp\{ \sum_{j=1}^p X_{ip} \beta_p   +   \xi_i \} =
$$
для таких $i$ не будет изменяться. при этом, для наблюдений $i$, относящихся к уровню $L$ соотвествующий коэффицент $X_{ij} = 1$. тогда выражение
$$ 
exp\{ \sum_{j=1}^p X_{ip} \beta_p   +   \xi_i \} =
$$
уменьшится, при уменьшении $\beta_j$.

Доказанное утверждение, означает, что в таком случае не существует оптимального вектора $\beta_j$. Нестрого говоря, для таких данных оптимальное значение $\beta_j = - \infty $. На практике, это приведёт к тому, что численный алгоритм, нахождения $\beta$ в какой-то момент остановиться и сделает $\beta_j$  большим отрицательным, что приводит к странным искажённым результатам.

Данные рассуждения показывают, почему каждый уровень любого фактора должен содержать достаточно большое количество факторов. Эта проблема затрагивает многоуровневые факторых, речь о которых пойдёт ниже.







\subsection{Структура ОЛМ}

Структура ОЛМ может быть описана, как:
\[
	\mu_i = E[Y_i] = g^{-1}(\sum_{j}{X_{ij}} \beta_{j} + \xi_{i})		
\]

\[
	Var[Y_i] = \frac{\phi V(\mu_i)}{\omega_i}
\]

где

\begin{itemize}
	\item $Y$ вектор наблюдений
	\item $g(x)$ функция связи - заданая обратимая функия, которая связывает отклик с линейной комбинацией признаков.
	\item $X$ - матрица плана, полученная из факторов
	\item $\beta$ - вектор параметров модели, которые будут оцениваться.
	\item $\xi$ - вектор известных смещений.
	\item $\phi$ - параметр для масштабирования функции $V(x)$
	\item $V(x)$ - функция дисперсии
	\item $w$ - вектор весов, которые задают достоверность данных для каждого наблюдения.	
\end{itemize}

Линейная комбинация объясняющих переменных $\eta = X \beta$ называется \textit{систематической компонентой модели} или \textit{предиктором}, а сами элементы вектора $Y$ называются \textit{случайной компонентой модели}

Вектор откликов, матрица плана и смещения, основываются на данных. Функция связи $g(x)$, функция дисперсии $V(x)$ и параметр $\phi$ задаются исследователем.



\subsubsection{функция связи}

Функция связи показывает взаимосвязь случайной и систематической компоненты модели. Выбирая функцию связи, мы можем менять вид влияния переменных на переменную отклика. Например, взяв $g(x)$ тождественной,мы получим модель, в которой каждый фактор оказывает аддитивное влияние на наблюдаемую переменную. 

выбором $g(x)$ мы можем добиться так же и мультипликативного влияния. для  эгого нужно взять $g(x) = ln(x)$ 
тогда получим:

$$ E[Y] = g^{-1}(X \beta) $$

$$ E[Y] = exp\{X \beta\} $$

$$ E[Y] = exp\{ \sum_{i = 1}^pX_i \beta_i \}$$

$$ E[Y] = \prod_{i = 1}^p exp\{ X_i \beta_i \}$$

Варьируя $\beta_i$, мы можем произвольно изменять $exp\{ X_i \beta_i \}$, а эта величина имеет мультипликативное влияние на $ E[Y_i] $. При тарификации КАСКО используется именно эта функция связи, поскольку мультипликативная зависимость легко интерпретируется, что позволяет добиться высокого качества модели. Модели с такой функцией связи называются \textit{мультипликативными}.


\subsection{Стандартные ОЛМ модели}

Как было указано выше, для анализа среднего убытка для заданного риска, полезно разбить этот риск на две составляющие: частоту убытков и их тяжесть, а затем просто перемножить. После этого стоит задуматься о предположениях для этих моделей. При анализе частоты убытка стандартной является мультипликативная модель с Пуассоновским распределением. Такая модель замечательна тем, что она даёт результаты инвариантные относительно изменения временного периода наблюдения. То есть при изменении длины рассматривомого промежутка скажем с 1 года до двух результаты не изменяться. Для некоторых других распределений (например, гамма) это свойство не выполняется.


Для модели тяжести стандартной является мультипликативная модель с гамма распределением. Эта модель удовлетворяет свойству инвариатности относительно изменения валюты. То есть при моделировании с валютой в рублях или в долларах, мы получим одинаковый результат. Такое свойство выполняется не для всех распределений, например, для Пуассоновского.


