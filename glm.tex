
\subsection{Введение в ОЛМ}


Назначение обобщенных линейных моделей (ОЛМ) показать зависимость между наблюдаемой величиной, также называемой откликом $Y$ и зависимыми переменными, называемыми признаками $X$. Модель рассматривает наблюдения $Y_i$, как реализации случайной величины $Y$.

%translate_problem covariate - признак



\subsection{Структура ОЛМ}

Структура ОЛМ может быть описана, как:
\[
	\mu_i = E[Y_i] = g^{-1}(\sum_{j}{X_{ij}} \beta_{j} + \xi_{i})		
\]

\[
	Var[Y_i] = \frac{\phi V(\mu_i)}{\omega_i}
\]

где

\begin{itemize}
	\item $Y$ вектор наблюдений 
	\item $g(x)$ функция связи - заданая обратимая функия, которая связывает отклик с линейной комбинацией признаков.
	%translate_problem design matrix - проекционная матрица
	\item $X$ - проекционная матрица, полученная из факторов
	\item $\beta$ - вектор параметров модели, которые будут оцениваться.
	\item $\xi$ - вектор известных смещений.
	\item $\phi$ - параметр для масштабирования функции $V(x)$
	\item $V(x)$ - функция дисперсии
	\item $w$ - вектор весов, которые задают достоверность данных для каждого наблюдения.	
\end{itemize}

Вектор откликов, проекционная матрица и смещения, основываются на данных. Функция связи $g(x)$, функция дисперсии $V(x)$ и параметр $\phi$ задаются исследователем.
